\chapter {---¡Estoy casi harto de la Proporción Áurea}
Hemos encontrado casi hasta el hartazgo información referida a la proporción áurea en general y de ella en relación a la música en particular, y hemos encontrado casi hasta el hartazgo que en dicha información está presente casi hasta el hartazgo la idea de que tal proporción se encuentra casi hasta el hartazgo en la naturaleza principalmente por una razón de eficiencia y utilidad: es eficiente y útil para el árbol disponer según esta proporción sus ramas para que el sol llegue a muchas de sus hojas; es eficiente y útil para el caracol tener forma de caracol y esta forma responde a \emph{esa} proporción; es eficiente y útil para la estrella de mar ser una estrella con \emph{esas} proporciones\ldots
Podría decirse, por otro lado, que nuestros sentidos \emph{se educan}, por percibir toda esa naturaleza de la que parte somos, conociendo como \emph{bueno} aquello que responde a la proporción áurea, ya que es \emph{bueno} ser naturalmente eficiente porque serlo significa aumentar las posibilidades de supervivencia, lo cual es \emph{bueno}. Bueno.


\section{Casi hasta el hartazgo en general}

---Si divido una recta en dos ¿cuántos elementos tenemos?\\
---Dos.\\
---No, tres.\\
---¿Por qué?\\
---Porque obtengo la parte A, la parte B y el todo\ldots\\
---¡Aaaah!\\
---¿Y cómo hacer para relacionar, desde el punto de vista de las proporciones
entre los elementos, A, B y el todo?\\
---Eeeeh\ldots\\
---¡Sí, con $\Phi$!\\
---¿$\Phi$?\\
---¡Sí, $\Phi$, Phi!\\
---Fi\\
---Sí, Phi, la letra griega que representa un número irracional que se obtiene así: $$\Phi=\frac{1+\sqrt{5}}{2}=1,618033989...$$\\
---¿Y para qué sirve?\\
---Para nada, $\Phi$ es algo absolutamente ineficiente e inútil pero al principio parece todo lo contrario. Representa el cociente de la \emph{proporción áurea}, la cual es ni más ni menos que la manera de relacionar los tres elementos de los que estábamos hablando: el segmento que llamamos A, el otro segmento B y toda la recta.\\
---¿Podría darme más información, por favor?\\
---Cómo no, puedo darle información casi hasta el hartazgo. Le pongo un ejemplo: si a la recta la dividimos en dos partes de forma que la parte A y la parte B midan exactamente lo mismo, habrá entre esas partes una relación de \emph{igualdad} que podríamos expresar como proporción 1:1, pero ninguna de estas partes mantiene la misma relación con el todo, ya que él es el \emph{doble} de cada una de ellas, y esa proporción se expresa como 2:1; tenemos entonces por un lado una proporción 1:1 y por otro una proporción 2:1 y no logramos establecer una única relación entre los tres elementos: A, B y el todo. En cambio la \emph{proporción áurea} es aquélla que divide al todo en dos partes \emph{desiguales}, y que la parte mayor y la parte menor mantienen una proporción que es exactamente la misma que mantienen el todo con la parte mayor. Así se logra relacionar los tres elementos: A, B y el todo.\\
---¡Aaaah!\\
---$\Phi$, como ya dije, es un cociente que surge de la división
entre las dos partes entre sí o del todo con la parte mayor. Dicho
de otro modo, siempre que dividamos los valores de dos partes relacionadas y el cociente dé como resultado $\Phi$, esas partes estarán en \emph{proporción áurea}.\\
---¡Ooooh!\\
---Como se encuentra esta proporción en muchos seres de la naturaleza y muchos seres de la naturaleza atribuyen ésta a una creación de Dios, esa proporción tomó el nombre que tiene: \emph{proporción áurea} o \emph{proporción divina}.\\
---¡Uuuuh!\\
---De ahí que muchos artistas tomaron esta especie de regla matemática para construir según ella sus obras. Pintores, arquitectos, escultores y músicos estuvieron por siglos prestándole atención al irracional $\Phi$. $\Phi$, ese inútil, si algo útil tiene es que si multiplicamos el valor del \emph{todo} por la parte decimal de $\Phi$, es decir 0,618033989\ldots obtenemos el punto áureo por el cuál dividir el todo en dos partes divinamente, naturalmente proporcionadas. ¿Sigo?\\
---Y\ldots

\section{Casi hasta el hartazgo en particular}
El sistema tonal, y la mayoría de los sistemas de organización de las alturas de los sonidos, basa su existencia en un axioma: el intervalo de octava es el ciclo para el oído humano. A la vez, tal sistema divide a ese intervalo en doce partes iguales, los semitonos. Para dividir a la octava en proporción áurea dentro del sistema tonal podemos aplicar la fórmula del siguiente modo: $$12*0,618033989=7,416407868$$
Como no existe en el sistema 7,4 semitonos decimos que el punto áureo de la octava está en los 7 semitonos (quinta justa), por lo que es posible dividirla de estas dos maneras:

\begin{figure}[H]
\begin{center}
\begin{tabular}{c|c}
\textsc{Parte mayor en el grave} & \textsc{Parte mayor en el agudo}\\
\hline \\
\lilypond[notime]{ \relative {<c' c'>1 <c g' c>}}

&
\lilypond[notime]{ \relative { <c' c'>1 <c f c'> }}
\end{tabular}
\end{center}
\caption{\emph{La octava dividida en proporción áurea.}}
\end{figure}

Si a la vez buscamos la proporción áurea de la quinta (7 semitonos) obtenemos la tercera mayor (4 semitonos):
$$7*0,618033989=4,326237923$$

\begin{figure}[H]
\begin{center}
\begin{tabular}{c|c}
\textsc{Parte mayor en el grave} & \textsc{Parte mayor en el agudo}\\
\hline \\
\lilypond[notime] { \relative {<c' g'>1 <c e g>}}
&
\lilypond[notime]{ \relative {<f' c'>1 <f aes c>}}
\end{tabular}
\end{center}
\caption{\emph{Proporción áurea del intervalo de quinta justa.}}
\end{figure}

Hemos llegado a obtener, a través de dividir la octava y la quinta
en proporción áurea, las dos configuraciones sonoras simultáneas fundacionales de la armonía del sistema tonal: los acordes mayor y menor.

Así como en una recta tenemos dos puntos áureos (dividiéndola en mayor-menor y menor-mayor), en la octava, recta musical, también (ver Ejemplo \ref{ej5}).

\begin{figure}[H]
\begin{center}
\lilypond[notime]{ \relative {c''1 g f c}}
\end{center}
\caption{\emph{Los dos puntos áureos dentro de la octava.}}
\label{ej5}
\end{figure}

Vista así la octava ya nos hace pensar en los tetracordios de los
antiguos griegos y la manera de éstos de construir escalas, con la simetría como parámetro primordial. Si a la vez producimos un punto áureo entre Do5-Sol4 y otro entre Fa4-Do4 ($5*0,618033989=3,090169944$), buscando la simetría entre las dos cuartas disjuntas contenidas en la octava, llegamos, oh sorpresa, a una escala pentáfona:

\begin{figure}[H]
\begin{center}
\lilypond[notime]{ \relative {\[ c''1 bes g \] \[ f ees c \]}}
\end{center}
\caption{\emph{Escala pentáfona, simétrica y áurea.}}
\end{figure}

Una melodía construida sobre esta escala, que busque ubicar en sus momentos más importantes (inicio y fin de frase) las notas que proporcionan a la octava en forma áurea, estará aportando una organización formal asociada a tal proporción: (Ejemplo \ref{ej7})

\begin{figure}[H]
\begin{center}
\begin{lilypond}
\relative g' {
  \tempo "Largo de acá (¡hay que seguir defendiendo lo nuestro!)"
  \partial 4 \[ c8( bes
  g2) c8( bes g f
  g2.) \] \[ f8( ees
  c2) f8([ ees c bes]
  \partial 2. c2.) \] \bar "|."
}
\end{lilypond}
\end{center}
\caption{\emph{Melodía simétrica basada en la escala pentáfona áurea.}}
\label{ej7}
\end{figure}

Observamos que en la primera frase, marcada en el ejemplo con un corchete, tiene el equivalente a 16 corcheas en duración. ¿Proporcionamos?
$$16*0,618033989=9,888543824$$
Así que el equivalente a 6 y 10 corcheas corresponden a las partes menor y mayor de 16, justo la duración de la primera y segunda hemifrases, marcadas en el ejemplo con ligaduras. La segunda frase no es más que la réplica de la primera, un tetracordio más grave.

Aspectos armónicos, melódicos, rítmicos y formales danzando alrededor de la proporción áurea en el sistema tonal (y en otros). Quien use este sistema aunque nunca haya siquiera oído hablar de la proporción áurea, estará usándola.

\section{Casi harto de la tonalidad en particular y de la naturaleza en general}
---Dígame, usted que es músico, ¿por qué siempre se dice que Mozart, Bach, Beethoven, Debussy, Bartók y otros usaban en su música la proporción áurea?\\
---Porque la usaban. Todos ellos usaban el sistema tonal (en el caso de Bartók es una \emph{tonalidad ampliada} en varios sentidos pero que no deja de estar basamentada en la tonalidad más tradicional), y el sistema tonal aparenta estar construido sobre la proporción áurea.\\
---Disculpe que lo interrumpa, pero hay una palabra en su discurso que me hace un poco de ruido: \emph{aparenta}. ¿El sistema tonal \emph{no es} acaso construido sobre la proporción áurea?\\
---Buena observación. No puedo más que confirmar sus sospechas. El sistema tonal divide a la octava en 12 semitonos iguales tras una larga caminata cultural que fue pasando por diversos sistemas de afinación y muchos tipos de temperamentos, todos ellos producto de las necesidades compositivas de cada época y ninguno de ellos respondiendo a la naturaleza. En el igual temperamento, salvo la octava, nada está de acuerdo a la naturaleza del sonido, nada está, digámoslo así, afinado. Por eso tomar como unidad de medida de la octava el semitono igual temperado es ya tomar una unidad de medida no-natural, surgida de prácticas culturales de siglos, así que la proporción áurea de la octava deducida desde esos semitonos \emph{no es} la proporción áurea de la octava, sino la proporción áurea del sistema tonal.\\
---Comprendo, creo\ldots\\
---Aclarado que la tonalidad no representa a la naturaleza, me interesa decir algo más.\\
---Usted dirá.\\
---Que creer que la proporción áurea encierra a toda la naturaleza es por lo menos inocente; que creer que la proporción áurea es la única posibilidad de la naturaleza es no haberse dado cuenta que la naturaleza tiene una abrumadora cantidad de posibilidades, y que algunas se manifiestan, otras se pierden y otras más quedan latentes. Pienso que es más pertinente hablar de \emph{naturalezas}, así en plural, en lugar de \emph{la} naturaleza.\\
---Creo saber hacia dónde va.\\
---¿Hacia dónde cree usted que voy?\\
---Hacia un pedido de auxilio.\\
---¡Sí, amigo, ha usted acertado! ¡Muerte a la proporción áurea! ¡Quiero ver y oír otras naturalezas! ¡Quiero que no nos resignemos a ver una sola posibilidad! ¡Quiero otras tonalidades, quiero otra afinación!\ldots Tal vez, después de todo, no quiera tanto la muerte de la proporción áurea. ¿Qué le parece?\\
---Y\ldots
