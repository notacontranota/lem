\chapter {Comala, el micrófono y orquestas sinfónicas}
\section*{\centering \begin{normalsize}Resumen\end{normalsize}}
\begin{quotation}
\noindent El Sistema de Orquestas Infantiles y Juveniles de la provincia de Salta (norte de Argentina) nace en 2005 tras la creación de la orquesta sinfónica de la misma provincia (2001). Ambas instituciones poseen antecedentes que legitiman su presencia oficial, pero es el surgimiento de estas entidades estatales las que marcan un antes y un después en la vida musical de esta región marginal de un país marginal. A pesar de reconocer un antes y un después, también reconocemos una línea que atraviesa ambos períodos, línea trazada con elementos de pobreza estructural, dependencia económica y cultural, autoritarismos ancestrales y contemporáneos, y también algunas riquezas. Las políticas culturales y educativas implementadas en las orquestas sinfónicas de Salta (repertorio, inclusión y exclusión social) son un síntoma que en estas páginas nos proponemos leer, y a partir de esta interpretación aventurar también algún pronóstico.
\end{quotation}
%\clearpage
\vspace{1cm}

\begin {quotation}
\begin {flushright}
\begin {minipage}{6cm}
\emph{El sol le llegaba por la espalda. Ese sol recién salido, casi frío, desfigurado por el polvo de la tierra.}
\begin{flushright}
\textsc{Juan Rulfo}
\end{flushright}
\end {minipage}
\end {flushright}
\end {quotation}

\section{Antes, después y ahora}

Antes y después.

Antes había una música del lugar, antes había poesía, antes había pinturas. Antes no había ballet, antes no había orquesta sinfónica. Después hubo otra música del lugar, después hubo otra poesía, después hubo otras pinturas. Después hubo un ballet clásico y un ballet folclórico, después hubo orquestas sinfónicas (profesional e infanto-juveniles).

¿Antes y después de qué?

Un poco de historia argentina del siglo XX permitirá visualizar el
proceso que nos lleva a ese momento de inflexión que marca un antes y un después en el país y en la provincia de Salta.

El partido político más importante durante el siglo XX en Argentina fue el partido militar. El Ejército Argentino, desde 1930 fue un actor central en la vida política e institucional del país. Ya aquel primer golpe de estado abrió las puertas a un proceso político que tuvo como ejes conductores a la corrupción, el fraude electoral y la exclusión y empobrecimiento de las clases trabajadoras. En 1943, un sector del mismo ejército produce un nuevo golpe de estado, esta vez intentando revertir la dirección tomada por el anterior. De este segundo golpe de estado nace el más importante movimiento político de la historia argentina: el Peronismo. Perón, amplio ganador de las primeras elecciones presidenciales limpias en más de 17 años, gobernará el país por poco menos de 10 años, hasta que el sector opuesto al peronismo dentro del ejército resolvió llevar adelante un nuevo golpe de estado. La autodenominada Revolución Libertadora fusiló gran cantidad de militares afines a Perón y dirigentes políticos, y proscribió al partido peronista. Esta proscripción duró 18 años, por lo que las pocas elecciones democráticas que hubo entre 1957 y 1973 contenían el vicio de la ausencia de la representación política mayoritaria de Argentina. Así y todo, en 1963 hubo un gobierno, el del doctor Illia, cuyas políticas socio--económicas eran muy similares a las de Juan Perón. Un nuevo golpe de estado, en 1966, dio fin a ese proceso político, y esta vez con una novedad. La represión política a través de la violencia física no era una novedad, aunque en el gobierno de Onganía (1966-1969) ésta se dio en un grado hasta entonces no vivido. Lo que sí fue novedad fue el ataque directo a la educación pública, especialmente a la educación superior (universitaria). Tropezando con sus propios desaciertos, y tras una sucesión de mandos siempre militares en la presidencia, la dictadura llama finalmente a elecciones para 1973, y esta vez sin la proscripción del peronismo. Para ese entonces, con 18 años que fomentaron el mito de Perón y el odio entre los sectores de izquierda y de derecha del movimiento, la sociedad argentina sufría un proceso de fuerte enfrentamiento. Perón, anciano y enfermo, asume el poder ganando muy holgadamente las elecciones, resultado que en gran medida se explica porque siempre se le reconoció al líder del movimiento justicialista su habilidad para hacer convivir a opuestos en forma armoniosa. A los pocos meses, en julio de 1974, Perón fallece. Decididamente, a partir de entonces, los sectores de derecha ganan terreno en el gobierno e inician una represión de los sectores de izquierda que involucra el secuestro, tortura y desaparición física de personas. 1976 es el año en el que el partido militar irrumpe nuevamente en el poder a punta de pistola. Esta vez son los hijos y sobrinos de los protagonistas del  olpe de 21 años atrás, y superando a la generación precedente llevará adelante la más sangrienta dictadura de toda la historia nacional, asesinando directamente a más de 30.000 personas por razones políticas y asesinando indirectamente a generaciones enteras al endeudar astronómicamente al país y al destruir sistemáticamente la industria nacional. 1983, tras la derrota militar en la guerra de Malvinas, da inicio a un proceso democrático que dura hasta la actualidad.

Salta, provincia argentina ubicada en el norte del país, que limita con seis provincias y tres países (Chile, Bolivia y Paraguay), en su autoconstrucción identitaria puso en una de sus bases a un héroe de la gesta independentista. Tal héroe, el general Güemes, fue crucial para el éxito de la estrategia general trazada por el general San Martín. Sin su campaña militar la independencia de las colonias españolas de la América del Sur no hubiese sido posible al menos en ese tiempo. El dato sobresaliente es cómo solventa Güemes esa campaña militar. El general Martín Güemes, miembro de una familia socialmente bien posicionada, llega a la gobernación de la provincia de Salta,   obliga a las familias más pudientes a pagar el impuesto para la guerra. El odio de la oligarquía local hacia Güemes llega al extremo de traicionarlo y entregarlo al enemigo, quien no duda en matarlo. Esa misma clase social, en gran medida autora de la construcción de identidad cultural de Salta ---quien tiene el micrófono habla más fuerte---, se apropia de la figura de Güemes presentándolo como \emph{su} héroe a la vez que provoca un olvido sobre las políticas socio-económicas llevadas adelante por el general durante su gobernación.

Salta, provincia con amnesia, tuvo entre 1973 y 1975 un gobernador que ponía en práctica políticas comparables a las conducidas en su momento por el general Güemes. La policía provincial intentó hacer un golpe de estado, que fue repelido por gente del pueblo, e inclusive reos, convictos que llevados por el leal jefe de penitenciaría defendieron al gobernador doctor Ragone y a la institucionalidad democrática. Poco después el ideológicamente opuesto gobierno nacional intervino la provincia, haciendo así efectivo el golpe de estado. No pasó mucho tiempo de eso y Ragone es asesinado por la derecha que ya llevaba adelante la represión ilegal desde el estado nacional. Actualmente hay un proceso judicial por el homicidio de Ragone, y testigos muy importantes incriminaron directamente a quien fuera a la postre gobernador de la provincia: Roberto Romero.

Antes de la dictadura militar de 1976-1983 había una música del lugar, poesía, pinturas\@.\@.\@. El inmediato después de la dictadura muestra a Salta como un lugar muy parecido a Comala, el pueblo de \emph{Pedro Páramo}, novela de Juan Rulfo de cuyo texto tomamos las palabras que sirven de epígrafe a esta comunicación. Desierto, ahogo, muerte, palabras que bien califican el inmediato después, el después de los pintores, poetas y músicos.

Antes, en 1968, un «extranjero», José Alberto Sutti, originario de la ciudad de Rosario de Santa Fe, que está ubicada en una de las provincias más ricas del país, habiéndose radicado en una de las provincias más pobres del país decide fundar una orquesta para tocar música clásica. Un año después la agrupación logra el apoyo para bajos sueldos para los músicos de la intendencia de la ciudad y se convierte en la Orquesta Municipal de Cámara de Salta, que actuará hasta su desaparición en octubre de 2000. Previo a esa muerte institucional, y durante el mismo año, las autoridades municipales prohibieron el ingreso de Sutti al teatro de la ciudad, lugar donde ensayaba la orquesta, para evitar que dicho organismo siguiera actuando.

Antes, en las décadas de los años '30, '40, '50, una generación de pintores, en su mayoría «extranjeros» de otras provincias, desarrollaron su arte en alto nivel. Ahora Gertrudis Chale, Raúl Brié, Carybé y Luis Pretti son homenajeados en el museo provincial de bellas artes mientras esta sociedad expulsaba, exiliaba a esos artistas cuando en vida.

La música y la poesía que durante los años '40, '50 y '60 se venía desarrollando fue acallada durante la última dictadura militar. Sin embargo tratemos de no olvidar que la amnésica Comala ya era Comala antes de convertirse en Comala, matando a su máximo héroe, matando a su mejor gobernador, matando a sus mejores pintores, músicos y poetas.

Después, en 1984, tras la vuelta a la democracia, Roberto Romero, directo beneficiado de la muerte de Ragone, ya en la gobernación de la provincia funda la denominada Orquesta Estable de la Provincia de Salta, que, músicos más, músicos menos, clonaba, salvo por el director, a la Orquesta Municipal. No había más músicos que esos treinta o cuarenta que integraban las orquestas mellizas que pagaban medios sueldos.

Antes, al principio de la década de los años '70, se funda la Escuela Superior de Música ---José Sutti es co-fundador de dicha escuela---y la Universidad Nacional de Salta. De la escuela de música no salen músicos sino profesores que se dedican a enseñar en la misma escuela que sigue sin producir músicos (y en la orquesta con dos cabezas seguían envejeciendo los mismos 30 músicos), y en la universidad, igual. Sin escuelas de arte y sin escuelas de ciencia esta sociedad seguía dependiendo de arrebatos individuales y de «extranjeros» aventurados a radicarse en un lugar que por suerte no tiene el clima de la Comala de Rulfo, sino todo lo contrario. Ahora, la escuela de música y la universidad no actúan de modo muy diferente a entonces: persisten en su enviciado aislamiento social y en su ineficiencia académica.

Después de Roberto Romero, y apadrinado por él, lo sucede en el poder un hijo de las antiguas familias gobernantes de Salta, de esas familias que ya estaban antes y después de Güemes, y ahora también. Y a ese gobernador le siguió ---¡atención!--- quien fuera gobernador de facto durante la dictadura militar. Y le siguió ---¡más atención!--- el hijo de Roberto Romero, quien reformó la constitución provincial para quedarse durante 12 años  en la gobernación (1995-2007). Él, Juan Carlos Romero, quien en sus políticas socio-económicas adhiere fuertemente al modelo neoliberal llevado adelante por la dictadura militar y por el contemporáneo gobierno de Carlos Menem durante la década de los '90 (1989-1999), es quien además lleva adelante una política cultural tendiente a afianzar a Salta como una heredera de la cultura occidental, y para ello funda un ballet clásico y una orquesta sinfónica (en el proceso desaparecen las mellizas mutadas), dos museos (bellas artes y contemporáneo), remodela un teatro (aunque en el medio cierra otro) y llena de lucecitas el centro de la ciudad, al tiempo que la provincia logra los máximos históricos en pobreza estructural, desnutrición y mortalidad infantil, deserción escolar, analfabetismo, y un largo y triste etcétera. Hacer una orquesta sinfónica de un día para otro en una provincia sin músicos sinfónicos condujo las cosas por el único camino viable: importar músicos. «Extranjeros» y extranjeros poblaron la agrupación musical. Desde su fundación (2001) hasta la fecha la orquesta tuvo tres directores titulares: el extranjero Felipe Izcaray (que Comala ya mató), el «extranjero» Luis Górelik (que Comala ya mató) y al peor director de la historia, una figura casi fantasmal (que Comala no se animó a matar, y se suicidó). Hoy, con un director interino que fue subdirector todos estos años, esperamos, entre otras cosas, saber cómo será su inevitable muerte institucional (en Comala sólo los fantasmas no mueren de muerte violenta).

Al haber una orquesta importada y una escuela de música fantasma, la necesidad de lograr un genuino desarrollo en la interpretación de la música sinfónica en Salta se puso de manifiesto. La necesidad de orquestas infantiles y juveniles hizo inevitable el nacimiento de una nueva institución educativa: el Sistema de Orquestas Infantiles y Juveniles de Salta, que nace en 2002 de la mano de su fundadora Kelly Wayar, y que consigue a través del venezolano Felipe Izcaray (por entonces director de la orquesta sinfónica) apoyo oficial a partir de 2006. Tres son los más importantes antecedentes históricos en cuanto a orquestas infanto-juveniles en Salta, que legitiman este proyecto más allá de la necesidad surgida con la orquesta sinfónica: la Orquesta Juvenil que dirigieran José Aguirre y Mónica Saborida y que carecía de presupuesto y de profesores (llena de buenas intenciones y vacía de recursos materiales y humanos); los ensambles que en la escuela de música dirigía Ítalo Petracchini, quien estaba condenado a andar persiguiendo alumnos para que vayan a los ensayos; Filarmúsica (1991-1994), agrupación de instrumentos de cuerdas y un clave (órgano eléctrico japonés haciendo las veces de) dirigido por Juan Ramón Jiménez, que tuvo éxito en lo musical y en lo educativo (había un profesor de violín que a su vez les daba ayuda a los violistas, y uno de cello, que era el director), pero al carecer de recursos económicos el agotamiento que produce ``tirar el centro y cabecear'' no tardó en hacerse notar, provocando el deceso de la agrupación. A partir de 2007, el gobierno nacional, a cargo de Néstor Kirchner, empieza a implementar en todo el país un sistema de orquestas infantiles y juveniles similar al organizado hace más de 40 años en la imitable Venezuela. Hoy, desde entonces, conviven, sin demasiada articulación entre ellos, los proyectos nacional y provincial. El proyecto provincial, con más tiempo de funcionamiento y con mejor presupuesto que la implementación local del proyecto nacional, ha logrado ya abastecer con algunos músicos a la orquesta sinfónica de Salta, por lo que su principal razón de ser se ve justificada en los resultados. Por primera vez en la historia sinfónica de Salta surgen músicos de calidad que pueden quedarse trabajando en la provincia.


\section{Inclusión, exclusión y repertorio}

La otra razón de ser de las orquestas sinfónicas infantiles y juveniles de Salta es la necesidad de contener socialmente a muchas personas marginadas y empobrecidas por tantos años de desgobierno sistematizado y liderazgos sociales descabezados. La presencia de la última dictadura no fue sino en el contexto de un plan general para la América del Sur denominado Plan Cóndor (la ocupación norteamericana del continente a manos de los ejércitos locales a los fines de evitar una nueva Cuba y seguir depredando la región \emph{a piacere}), plan que fue exitoso sobre todo en sus objetivos de ruptura cultural. La pobreza instalada, más que material fue cultural. El embrutecimiento de la población es la condición previa a la colonización cultural. En tal colonización los medios de comunicación cumplieron y cumplen un rol central: quien tiene el micrófono dice más fuerte que los demás, y consecuentemente lidera. Lidera opinión, lidera gustos (se gusta de lo que se es capaz de reconocer, nos advierte T. Adorno), lidera políticas públicas. En un mundo donde la industria discográfica dicta a los medios qué difundir según un criterio exclusivamente económico es de esperar que los bienes culturales en cuanto a música se refiere se distribuyan mucho peor que los bienes materiales. Si en el mundo es así, en una villa miseria de Salta con mayor razón. El acceso a música sinfónica, esa música que desde la gobernación se anheló heredar, es, en la mayoría de la población, nulo o casi nulo. El oído colonizado del pueblo parece haber cobrado inmunidad a Beethoven, Schubert y Haydn, y parece haber cobrado alergia a Stravinsky, Bartók y Berg. ¿Pero es así? ¿Cómo saber si hay inmunidad o alergia a algo completamente ausente? Los niños que ingresan a las orquestas infantiles tocan y gustan de la música clásica porque empiezan a conocerla y a reconocerla. Drogadicción, delincuencia, vandalismo no son más que síntomas de una o de muchas ausencias. La presencia de la música sinfónica en niños de cualquier clase social los aleja de tales síntomas. Al menos en la experiencia salteña se puede verificar. El sistema de orquestas de la Provincia ha sido capaz de captar más a personas de sectores socioeconómicos medios que a personas de clase baja, aunque compensa haciendo una labor muy plausible con niños enfermos oncológicos y personas con síndrome de Down. En el sistema de orquestas de la Nación supieron captar más a las clases sociales más desprotegidas, y es ahí donde se observa claramente cómo la música presente ausenta miserias. A pesar de la comprobación cotidiana de la aceptación de la música sinfónica por parte del joven oído colonizado, el prejuicio de profesores y directores de las orquestas hace que, «a los fines de ir acercando de a poco» a los estudiantes a ese mundo que siempre esquivo les fue, el repertorio que se aborda incluya en cantidades asombrosas la misma producción sonora gritada por el Gran Micrófono Amplificado. No podemos dejar de preguntarnos si ese tipo de decisiones se toman por razones pedagógicas, o porque los profesores y directores poseen no tan jóvenes oídos colonizados.

\section{Extraterrestres, pirámides y fermentos}

Antiguos Imperios supieron hacer pirámides y supieron hacer bebidas alcohólicas. Hoy, sus descendientes olvidaron o perdieron el conocimiento para hacer pirámides. No encontramos mejor ejemplo que este para mostrar cómo puede llegar a manifestarse el siempre latente peligro de pérdida de bienes culturales. Olvidamos cómo hacer pirámides porque dejamos de hacerlas. Está claro que no hemos olvidado cómo se hacen cervezas porque no hemos dejado de hacerlas porque no hemos dejado de tomarlas. La incapacidad de explicar un método de construcción de las pirámides, la pérdida de ese conocimiento, pone en escena a los extraterrestre: ¡ellos las hicieron!

Nuestras sociedades han sido también capaces de hacer tanto pirámides como fermentos musicales. Decir si la música sinfónica es mejor que otras músicas es comparable a decir si las pirámides son mejores que las bebidas alcohólicas, y no es esa la discusión que nos interesa. La pretensión de convertirnos en herederos de la cultura musical sinfónica (si no queremos eso, cerremos todas las orquesta ya mismo) no puede ir separada de una clara política conservadora de los bienes musicales sinfónicos: las orquestas sinfónicas son para tocar música sinfónica. Los avejentados y obturados oídos colonizados de la actual clase política salteña hizo y hace que en  estos diez años en general, y en los dos últimos años en particular (donde el fantasma de un director sobrevoló el teatro sin ópera), se haya utilizado a la orquesta, instrumento musical eficaz en la construcción arquitectónica de alta calidad, para fabricar chicha, aloja y vino patero. Las escuelas-orquestas, sin la obligación de la demagogia y sin la presión de generar espectáculo, igualmente caen en idéntica situación que la orquesta mayor. Oídos colonizados y modelo a seguir (la orquesta mayor) terminan pesando más que la prudencia de ejercitar e incrementar un conocimiento que no podemos darnos el lujo de perder. Imagino, por un momento, a un trabajador de las pirámides en el Imperio Azteca tomándose una cerveza (o tequila) en el descanso, e imagino, a cada rato, a otro trabajador de las mismas pirámides que bebía cerveza (y tequila) todo el tiempo, y por esa razón su fin no fue otro sino el sacrificio para los dioses que no tenían sed de cerveza (ni tequila). Si nuestros trabajadores de pirámides musicales y sus jefes siguen bebiendo  fermentos y olvidando y haciendo olvidar las pirámides musicales, tendremos que llamar a los extraterrestres para oír a Beethoven, Schubert, Haydn, Stravinsky, Bartók, Webern, y a Pedro Páramo para que enseñe composición musical y podamos  estrenar nuevas obras. A algunos funcionarios de alto rango podremos ofrecerlos a los dioses (que siguen con sed).