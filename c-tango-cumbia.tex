\chapter {Tango -- Cumbia}
Una amiga me invita a una fiesta. Estaba ella encargada de musicalizar la reunión y, como es de esperarse, lo hace desde su gusto personal: la banda sonora de la película \emph{Underground} se hizo presente esa noche y un comentario mío sobre una de las piezas ---«Parece una cumbia balcánica»--- desató la ira de la señorita: «Yo escucho otras escalas», «No me parece que esta música tenga que ver con
aquella otra» y un largo etcétera de gestos imposibles de acercar al lector mediante el lenguaje escrito. De vernos y hablarnos a diario pasamos a una ausencia de dos semanas. Sospeché entonces que, aunque sin querer y habiendo estado lejos de la premeditación, había yo tocado un punto de verdad que a mi amiga le provocó un estado de ofensa ---«la verdad ofende», recordé. Pero era esa una verdad a medias. Para terminar de ofenderla tuve que encontrar la otra mitad de la verdad: \emph{los géneros musicales no son sino dos, a saber: Tango y Cumbia}. Y esta afirmación no es una teoría, o al menos no lo es en el sentido occidental del término, que implica siempre una generalización, una abstracción y una conceptualización (a lo expuesto hoy acá le podemos llamar \emph{teoría} sólo en sentido amplio). Se acerca mucho más a un conocimiento expresado desde lejos del Método Científico y desde un lugar mucho más cercano al de las Culturas Originarias del continente americano, más amigas de lo empírico, de la metáfora y la analogía, y muchas veces más amigas de la verdad.

Todas las músicas caben en una de estas dos categorías (tango, cumbia). Entonces, en verdad, «a lo indio», con sólo ejemplos la idea se dará a entender: Bach, tango; Vivaldi, cumbia. Verdi, cumbia; Wagner, tango. Schönberg, tango; Stravinsky, cumbia. Mozart, salvo el \emph{Requiem}, cumbia. El bolero, cumbia. La cumbia, cumbia. La cumbia «villera», tango. Gardel, cumbia.

El «método» fue aplicado por fuera de la música con sorprendentes resultados. Literatura: Borges, tango; García--Márquez, cumbia. Países: China, cumbia; Japón, tango. Rusia, tango; Italia, cumbia (Uruguay puede generar algún conflicto\ldots{}). Deportes: box, cumbia; tenis, tango. Polo, tango; fútbol, cumbia.

Hay grandes espacios que se definen en general como tango o cumbia, pero que en su interior albergan subespecies de la misma o de otra naturaleza. Las religiones y el fútbol, por ejemplo, son casos generales cumbia. Sin embargo, el judaísmo es tango y el budismo es cumbia, aunque siempre dentro del contexto general. Boca Juniors es cumbia y Vélez Sarfield tango.  Messi, tango; Maradona, cumbia\ldots{}El fútbol, ya lo dijimos, cumbia.

Incluso la «teoría» puede ser objeto de sí misma y convertirse así en «metateoría», y decir que ella es cumbia.

Invito al lector a sumar a la escueta lista acá iniciada casos de tango y cumbia hallados en todo el espectro de la cutura humana.