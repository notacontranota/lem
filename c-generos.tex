\chapter {Los Géneros Musicales}
¿Frío/calor, o temperatura? «Frío/calor» representa una dicotomía, «temperatura» representa un fenómeno que engloba tanto a «frío» como a «calor». ¿Música culta/música popular, o música? ¿Sirve hablar de «culto/popular» como sirve hablar de «frío/calor»? Según quién hable y a quién hable. El servicio meteorológico no dice «hace mucho calor», sino «la temperatura es de 38º C». La verdulera de la esquina dice: «---¡Cómo está pegando la calor, no?». Un sociólogo, que en muchos aspectos es altamente equiparable a la verdulera de la esquina, puede, como a la verdulera «frío/calor», servirle y utilizar las categorías de «culto» y «popular», categorías que dan cuenta de un orden social como «frío/calor» de una sensación térmica.

La naturaleza, responsable en gran medida de la sensación térmica de los seres humanos, suele ser bastante más equitativa que las sociedades de los hombres, y por esta razón, aunque variable de individuo a individuo, dicha sensación térmica permanece constante dentro de un rango muy estrecho de cambios en la especie humana.

Las sociedades humanas, responsables en toda medida de la generación y distribución de bienes culturales, suelen ser bastante menos equitativas que la naturaleza, y por esta razón, aunque variable de sociedad a sociedad, la mayoría de dichos bienes culturales queda en posesión de una élite, dejando a una inmensa mayoría completamente marginada del más valioso conocimiento.

Un teórico musical o un músico, que en muchos aspectos es altamente equiparable al servicio meteorológico, puede, como al servicio meteorológico «temperatura», servirle y utilizar la categoría de «música», categoría que da cuenta del completo fenómeno que engloba a las organizaciones del mundo sonoro como «temperatura» de un completo síntoma climático.