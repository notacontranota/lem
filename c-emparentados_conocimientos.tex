\chapter{Los emparentados caminos del conocimiento}
En cualquier actividad humana hay posibilidades de búsqueda de conocimiento, de mayor profundidad y riqueza en el saber propio del campo que se estudia y en el que se actúa. El síntoma interesante es que, caminado un buen trecho de la senda elegida, comienza el caminante a percibir que caminos diferentes al elegido por él se entrecruzan con el suyo, y, al tenerlos a la vista, constata que ellos no distan demasiado en la fisonomía de la vía por la cual va transitando. Así, literatura e historia, artes marciales y danza, o música e informática ya no son lejanas entre sí, sino primas hermanas.

Pero en los caminos del conocimiento, como en las autopistas, se cobra peaje.

El conocimiento, como el dinero, es una de las cosas peor distribuidas de este mundo. Pero, a diferencia de quien tiene carencias materiales, quien es un pobre cultural las más de las veces desconoce serlo por no tener síntomas demasiado evidentes. «Se gusta de lo que se es capaz de reconocer» (Adorno), «Sólo se es capaz de re-conocer lo que se conoce» (un ser humano con pensamiento lógico), «Al pueblo eso no le gusta» (un demagogo), «¿Por qué? ¿Porque no conoce?» (un preguntón ruidoso).

El conocimiento, como el dinero, circula, y lo que circula en la sociedad a este nivel es lo que tanto individualmente como socialmente se fue y se va construyendo. Y como es fácil sospechar o constatar, no solamente llega a nosotros aquel conocimiento que es competencia de nuestro campo profesional. A la vez, nosotros generamos conocimiento que ponemos, por los canales que a nuestra disposición tenemos, a circular. Mayor circulación tendrá, entonces, aquel conocimiento que se transmita por canales más poderosos, canales a los que acceso tienen los individuos o los grupos de individuos más poderosos. Poder, conocimiento; conocimiento, poder.

Ya lo sabemos: pocos individuos de nuestra sociedad circulan libremente por los caminos del conocimiento, esos caminos capaces de llevarnos a regiones de una profundidad y una riqueza insospechadas. A esa altura de los caminos, y sólo allí, se empieza a vislumbrar el parentesco, la hermandad interdisciplinaria que más arriba hemos mencionado. Antes, todo parece inconexo, divorciado, ajeno. Antes de ese punto del camino, el riesgo de devenir mero operador en vez de persona civilizada es enormemente elevado.

Si nuestra intención es revertir este esado de cosas, en cuanto a la distribución del conocimiento se refiere, condición es convertirnos en individuos y grupos de individuos poderosos. Individuos y grupos de choque, grupos e individuos de vanguardia, dispuestos a pasar por encima de los puestos de control y de peaje, listos para quebrar las barreras prohibitivas y allanar los caminos. Así, y sólo así, literatura e informática, historia y danza, o artes marciales y música ya no son lejanas entre sí, sino primas hermanas.