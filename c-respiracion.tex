\chapter {La metáfora de la respiración}
Tras haber observado mínimamente tanto algunos fenómenos naturales como así también muchos fenómenos culturales, al menos hasta el momento, ninguno de ellos escapa a la posibilidad de compararlos o equipararlos al ciclo de la respiración, o mejor dicho a la acción de respirar: inhalar y exhalar, una carga y una consecuente descarga energética, cíclica y con variantes ocasionales que hacen de la respiración un fenómeno de infinitas posibilidades.

La música entendida por un lado como fenómeno cultural y por otro como metáfora de un cuerpo vivo no escapa tampoco a la posibilidad de ser igualada al proceso respiratorio. Así, una pieza musical, un cuerpo musical, vive, crece y se desarrolla y se oxida y muere de acuerdo a sus respiraciones musicales.
\begin{figure}[h]
\begin{center}
\begin{lilypond}
<<
	\relative {
		\repeat unfold 2 {g''8\rest g,16 c e g, c e}
		\repeat unfold 2 {g8\rest a,16 d f a, d f}
		\repeat unfold 2 {g8\rest g,16 d' f g, d' f}
		\repeat unfold 2 {g8\rest g,16 c e g, c e}
	} \\
	\relative {
		\repeat unfold 2 {b16\rest e8.~ e4}
		\repeat unfold 2 {b'16\rest d,8.~ d4}
		\repeat unfold 2 {b'16\rest d,8.~ d4}
		\repeat unfold 2 {b'16\rest e,8.~ e4}
	} \\
	{
		s1
		s1
		s1
		s1
	} \\
	\relative {
		c'2 c
		c c
		b b
		c c
	}
	>>
	\bar "|."
\end{lilypond}
\end{center}
\caption {\emph{Fragmento del Preludio 1 del \emph{Clave Bien Temperado I}, de J. S. Bac:h}}
\end{figure}

El primer proceso respiratorio del cuerpo musical llamado \emph{Preludio I de El clave bien temperado I de J. S. Bach} es una respiración normal, tranquila, y que ejemplifica acerca del movimiento de carga y descarga energética que en todo proceso respiratorio ocurre. Las armonías de los dos primeros compases van llenando de energizante oxígeno los pulmones del preludio, y en el tercer compás ya están ellos repletos de energía. Quedarse ahí envenenaría al cuerpo musical, por lo que necesario es abordar un proceso de liberación energética, que sucede hasta el final del cuarto compás. Y si decimos necesario decimos violento (toda necesidad es una violencia, nos dice Aristóteles en su Metafísica). Y si esa exhalación musical (movimiento armónico desde la dominante hacia la tónica) fue necesaria, ¿la inhalación no fue acaso igualmente necesaria para la vida del preludio? Toda esta reflexión sobre la violenta respiración musical bachiana nos recuerda al \emph{Cuaderno de Navegación} de Leopoldo Marechal, en particular al fragmento en el que el poeta argentino afirma que el de la vida no es un \emph{derecho}, sino un \emph{deber}. El Preludio de Bach no elige vivir, crecer, desarrollarse, oxidarse, morir; debe hacerlo. Y esta primera respiración dice mucho acerca de la salud presente y futura de uno de los más famosos hijos musicales del compositor alemán, hijo musical del cual difícil es poder afirmar que se trata de un asmático. Tampoco es un atleta de alto rendimiento del siglo XXI, o tal vez sí, pero no en competencia, sino en estado contemplativo, tranquilo, violentamente tranquilo (o necesariamente tranquilo).
