\chapter{Lo escrito y lo oral como constructor melódico}
La música en occidente se ha desarrollado de un modo único a partir de la adopción de la notación musical de \textsc{Guido D'Arezzo}. Ella es quien propicia una manera particular de pensar en música; ella es quien inaugura una cultura escrita de la música como no se encuentra en ningún otro lugar del mundo sino en Europa\footnote{En China hubo una escritura musical ideogramática, muy diferente a la notación \emph{gráfica} europea que permite «ver» la música en sucesión y en simultaneidad}. Se produce acá una bifurcación en el que hasta aquí había sido el camino de la música europea. Ahora existen dos caminos: uno central ---el de la música escrita---, y uno periférico ---el de la tradición oral.

\section[Prosa y verso]{Prosa y verso en la construcción melódica}

\section{La oralidad como motor de la melodía}
