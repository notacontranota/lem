\chapter {Pocos maestros, pocos discípulos}
He tenido en mi vida tres modelos de docente que me han marcado profundamente. Dos de ellos los encontré durante mi adolescencia, mientras el tercero llegó más tardíamente, a los 26 años de edad. Los dos primeros tienen nombre y apellido, mientras que el tercero ---aunque para mí lo tenga--- no.

A mis 13 años cursaba yo el primer años de Armonía en la escuela de música de Salta con una profesora que vivía por entonces sus últimos días. Tras su fallecimiento la suplió el \textsc{Dr. Oscar Rodríguez--Castillo}, quien con 20 años más que yo recién llegaba de EE. UU. Con muchas ganas de hacer muchas cosas y con poco o nada de diplomacia. Sus clases "extraprogramáticas" me cautivaron de entrada: Aristóteles en lugar de Rimsky--Korsakov, su «me enteré que los psico--pedagogos recomiendan no corregir con lápiz rojo» mientras entre sus dedos sostenía un lápiz rojo y a continuación solicitaba los trabajos en los que anotaba calificaciones como \emph{va queriendo} o \emph{¡Bien!}, sus ejercicios colectivos en el pizarrón, sus apuntes que con el tiempo se convirtieron en un libro de Armonía, sus conciertos en la Catedral de la ciudad hicieron del profesor un Maestro, capaz de exigir docenas de ejercicios para la semana siguiente y por lo tanto capaz de enseñar disciplina y trabajo como valores indispensables para la práctica de la escritura musical. Trabajar cansa, y eso a muchos ahuyenta.

Retomando el secundario ---que momentáneamente había suspendido por estudiar profundamente Contrapunto y Armonía con Rodríguez--Castillo--- conocí a la profesora \textsc{Zulma Palermo}, docente de la Cátedra de Lengua y Literatura, quien, más allá de haberme proporcionado enorme cantidad y calidad de contenidos, logró aportarme la forma de enseñar que en lo personal llegó a enamorarme. No éramos muchos los que llegamos a querer a esa profesora, pero eran muchos los que respondían positivamente en sus clases. Con posterioridad comprendí el por qué de aquéllo de «los pocos» y «los muchos». Ocurría que en esas clases, debido a la conducción ejercida por la profesora, estaba prácticamente prohibido no pensar, y lo que con los años descubrí es que pensar no es natural: hay que hacer fuerza para pensar, y eso a muchos ahuyenta.

Si hay un espacio en el que sobran las instituciones tradicionalistas y fuertemente conservadoras, ese espacio es el de las Artes Marciales. Tuve oportunidad de practicar una de ellas, de origen coreano, cuyo líder, el Maestro \textsc{Soo Nam Yoo}, fue y es muy celoso del modo de enseñanza del Arte. Entonces, no importa tanto quién enseña ---aunque a mí sí me importa mencionar, por haber sido mi primer Sabom, a \textsc{Maximiliano Álvarez}---, porque el Arte es el que siempre se enseña de la misma manera: la manera oriental, la manera de un niño, la manera de algunos occidentales: «observe e imite». Que no le expliquen con muchas palabras y que les pidan silencio y concentración, a muchos occidentales ahuyenta.

Tres «ahuyentadores» modelos de docente son los que me dieron forma. Alcanza con poner en situación de observar, trabajar y pensar, para «ahuyentar» a muchos. Y estos escritos surgen, en su mayoría, de mis ahuyentadoras clases a unos pocos occidentales que lejos estuvieron siempre de sentirse ahuyentados.